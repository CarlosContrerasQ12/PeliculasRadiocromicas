En este trabajó se encontró que las películas radiocrómicas son un instrumento útil para la garantía de calidad de procedimientos en radioterapia. Esto ya que proporcionan un método sencillo para realizar dosimetría relativa, permitiendo comparar las distribuciones de dosis entregadas realmente en un procedimiento de radioterapia con la planeadas en un principio.\\

Se estudiaron diversos fenómenos que se producen al trabajar con este tipo de películas, relacionados tanto a la parte de escaneo como a la parte de tratamiento digital posterior.\\

Todo el trabajo realizado es reproducible mediante el programa computacional diseñado y propuesto en los objetivos del proyecto. Este proporciona una interfaz amigable para el uso de estas películas en un ambiente clínico. Este posee las funcionalidades necesarias para realizar un proceso de calibración, generación de mapas de dosis y comparaciones mapa-plan que se requieren para asegurar la calidad del tratamiento.\\