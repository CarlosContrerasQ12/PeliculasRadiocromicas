La radioterapia se ha convertido en una de las armas modernas contra el cáncer más importantes por su efectividad como tratamiento para este. Este tipo de procedimientos consiste en la aplicación de determinada cantidad de radiación sobre el tejido enfermo para destruirlo, intentando conservar la mayor cantidad de tejido sano posible.\\

Aplicar una cantidad demasiado grande de radiación en tejido sano trae consecuencias que van desde síntomas pasajeros, como la perdida de cabello dificultad para tragar o incontinencia hasta consecuencias tan graves como la muerte\cite{cancer.net_2020}.  Por lo tanto, es necesario asegurar que en los procedimientos en radioterapia las cantidades y distribuciones de radiación estén bien controladas.\\

La aplicación de radiación ionizante para eliminar tejido enfermo es muy efectiva para destruirlo. Sin embargo, son grandes los requerimientos técnicos que se necesitan para asegurar que la radiación se concentre en su mayoría en el tejido enfermo y no en el tejido sano. La complejidad de los tratamientos ha aumentado mucho en diversos ámbitos, por lo que se hace aún más necesario establecer un control sobre el plan de irradiación que se está ejecutando.\\

Una de las maneras más usadas para ejercer este control de calidad es mediante el uso de películas radiocrómicas que, bajo un efectivo protocolo de medida, permiten obtener de manera experimental las distribuciones e intensidades de planes de tratamiento con radiación de múltiples tipos. Son una alternativa que permite obtener mayor resolución en la determinación de distribuciones de dosis en comparación con métodos como la dosimetría PORTAL con paneles de silicio, y que además presenta composición equivalente al tejido humano y es sencilla de utilizar para realizar comprobaciones de buena ejecución de tratamientos. \\

Por lo anterior, es necesario desarrollar buenas metodologías y herramientas que permitan usar estas películas de manera adecuada. Existen diversas clases de software comercial que realizan esta tarea, pero en general son costosos y no se dispone de uno en todos los casos. En consecuencia, el objetivo principal de este trabajo es desarrollar un sistema que permita el uso de este tipo de películas en el Centro de Control de Cáncer para fines de verificación dosimétrica. 