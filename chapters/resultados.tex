En esta sección se presentan resultados obtenidos para la calibración usada, así como la obtención y análisis de mapas de dosis obtenidos.
\section{Calibración}
Como fue mencionado anteriormente, se irradiaron películas con dosis que se registran en la tabla \ref{tab:DosisIrra}, con su correspondiente incertidumbre, medidas con la cámara de ionización.

\begin{table}[]
	\centering
	\begin{tabular}{|l|l|l|}
		\hline
		UM& Dosis(Gy)    & Incertidumbre($10^{-5}$) \\ \hline
		21.6&0.201472 & 3.8\\ \hline
		54&0.504137 & 5.23 \\ \hline
		108&1.00825  & 35\\ \hline
		216&2.01783  & 22\\ \hline
		432&4.037486 & 31 \\ \hline
		648&6.0583   & 48\\ \hline
		864&8.08168  & 142\\ \hline
		1080&10.1056  & 140\\ \hline
		1296&12.132   & 10\\ \hline
		1620&15.168   & 10\\ \hline
		2160&20.229   & 10\\ \hline
	\end{tabular}
	\caption{Dosis irradiadas}
	\label{tab:DosisIrra}
\end{table}

Las películas resultantes después de la irradiación se muestran en la figura \ref{fig:peliculasIrradiacion}. Todas las películas fueron escaneadas en la misma sección del escaner para reducir efectos de posición en la transmitancia.\\
\begin{figure}[H]
	\centering
	\includegraphics[width=0.7\linewidth]{images/peliculasIrradiadas.png}
	\caption{Películas EBT3 irradiadas con dosis entre 0 y 20 Gy }
	\label{fig:peliculasIrradiacion}
\end{figure}

Para relacionar determinados niveles de transmitancias medidas en una película con la dosis que produjo la coloración es necesario establecer una curva de calibración. En este caso se correlacionaron los promedios de transmitancia en regiones de interés de las tres películas irradiadas con la misma dosis con la dosis medida con la cámara de ionización que produjo dicha coloración. Con estos valores se establece la curva de calibración que se muestra en la figura \ref{fig:curvaFinal20} con sus respectivos parámetros.\\


\begin{figure}[H]
	\centering
\includegraphics[width=\linewidth]{images/calibracionMulti0a20.png}
	
	\caption{Curva de calibración con dosis de 0 a 20 Gy }
	\label{fig:curvaFinal20}
\end{figure}

En la cual se usó una curva del tipo 
\begin{equation}
D=\frac{AT+D}{T-C}.
\end{equation}\\

Según el manual del fabricante, las películas EBT3 son aptas para un rango inferior a 10 Gy, por lo que se prefieren omitir los últimos tres puntos de dosis, puesto que la película está en la región de saturación en esta región.  Además, estos puntos no aportan información adicional a los propósitos del trabajo, puesto que los planes que serán examinados no conllevan dosis tan altas. De esta manera se obtiene la curva que se muestra en la figura \ref{fig:curvaFinal}\\

\begin{figure}[H]
	\centering
	\includegraphics[width=\linewidth]{images/calibracionMulti.png}
	\caption{Curva de calibración con dosis de 0 a 10 Gy }
	\label{fig:curvaFinal}
\end{figure}

En esta se evidencia que las curvas ajustadas describen mejor los datos puesto que no se ha alcanzado la región de saturación, lo cual se demuestra con los mejores test de ajuste $\chi ^2$ calculados. También se evidencia que el canal que menos se ajusta y tiene un comportamiento más por fuera de lo esperado es el azul, consecuente con el hecho que en esta longitud de onda los cambios de coloración de la película son menores. Estas curvas son las que se usan a lo largo del trabajo para calcular mapas de dosis.\\

También se probaron más curvas sugeridas en la literatura, sin embargo, la que mejor se ajustó a los datos con las películas EBT3 en este rango de dosis fue la anteriormente mencionada. En la figura \ref{fig:CurvasAdicionales} se muestran otros dos ajustes con las respectivas formas de la función de ajuste y su bondad de ajuste.


\begin{figure}[H]
	\centering
	\subfloat[Curva cubica]{\includegraphics[width=0.7\textwidth]{images/calibracionCubica.png}\label{fig:cubic}}
	\hfill
	\subfloat[Curva exponencial]{\includegraphics[width=0.7\textwidth]{images/curvaCalibracionMalaExponencial.png}\label{fig:expo}}
	\caption{Curvas de calibración con otro tipo de funciones}
	\label{fig:CurvasAdicionales}
\end{figure}

Acá la curva \ref{fig:cubic} se ajustó a una función cúbica, pero no se pudieron estimar las incertidumbres en los coeficientes por una razón desconocida. Similarmente, en la curva \ref{fig:expo} se realizó un ajuste a una función exponencial que resulto de menor calidad comparado con la función racional usada inicialmente. Se debe notar que en este caso las curvas relacionan las cantidades  de densidad óptica con dosis absorbida y no de transmitancia directamente.\\

Finalmente, la única curva para realizar el  procedimiento de calibración multicanal fue la curva propuesta inicialmente, esto dado que las demás son susceptibles a errores numéricos que causan inestabilidad en el método de minimización usado.


\section{Efectos de diversos parámetros}

Para evaluar los efectos de diversos factores en la digitalización y posterior conversión de películas radicrómicas en mapas de dosis se estudiaron varios ejemplos. \\

En primer lugar, se estudió el efecto de la posición en el área de escaneo en la transmitancia medida con el escáner. Para esto se examina el comportamiento de la transmitancia de una película sin irradiar a lo largo de un perfil horizontal y vertical en esta. Con este objetivo, la digitalización de la placa que se muestra en la figura \ref{fig:fondoCero} se transformó en un mapa de dosis al cual se le extrajeron los perfiles horizontales y verticales que se muestran en la figura \ref{fig:perfiles}.\\

\begin{figure}[H]
	\centering
	\includegraphics[width=0.7\linewidth]{images/imagenFondoCero.png}
	\caption{Curva de calibración con dosis de 0 a 10 Gy }
	\label{fig:fondoCero}
\end{figure}

\begin{figure}[ht] 
	\centering
	\begin{minipage}[b]{0.6\linewidth}
		\centering
		\includegraphics[width=.5\linewidth]{images/imagenPerfilMapaCeroHorizontal.png} 
		\caption{Perfil horizontal} 
		\label{fig:perhor}
		\vspace{4ex}
	\end{minipage}%%
	\begin{minipage}[b]{0.6\linewidth}
		\centering
		\includegraphics[width=.5\linewidth]{images/imagenPerfilMapaCeroVertical.png} 
		\caption{Perfil vertical} 
		\vspace{4ex}
	\end{minipage} 
	\begin{minipage}[b]{0.6\linewidth}
		\centering
		\includegraphics[width=.5\linewidth]{images/perfilDosisCeroHorizontal.png} 
		\caption{Perfil de dosis horizontal} 
		\vspace{4ex}
	\end{minipage}%% 
	\begin{minipage}[b]{0.6\linewidth}
		\centering
		\includegraphics[width=.5\linewidth]{images/perfilDosisCeroVerticalEnCentro.png} 
		\caption{Perfil de dosis vertical} 
		\vspace{4ex}
	\end{minipage} 
\caption{Perfiles de dosis de película sin irradiar}
\label{fig:perfiles} 
\end{figure}

En estos perfiles se evidencia un sesgo sistemático en la dosis predicha para la película en cada punto. Se esperaría que, salvo variaciones muy pequeñas, los perfiles de dosis siempre dieran valores cercanos a cero. Sin embargo se observan varios comportamientos defectuosos reportados en la literatura. Se evidencia que en los tres canales de color se sobre-estima la dosis cuando el punto de estimación está lejos del eje central del escáner, es decir, cerca de los bordes.\\

También se evidencia que el canal azul proporciona medidas erradas de dosis por la poca sensibilidad que tiene la película en ese rango de dosis, lo que permite decir que no debe ser usado para otros objetivas más que la calibración multicanal.\\

Otra propiedad que permite evidenciar estos perfiles es la gran cantidad de ruido que mide el escáner, lo cual se ve las grandes variaciones locales en los perfiles de dosis. Esto es consecuencia de varios factores, el principal de ellos son los pixeles defectuosos que el escáner pueda tener. Para identificar estos pixeles defectuosos se cubrió el área de escaneo que se estaba usando con abundante cartulina negra, obteniendo la imagen escaneada que se muestra en la figura \ref{fig:fondoNegro}. En detalle se observan los pixeles absorben luz cuando no deberían. \\

\begin{figure}[H]
	\centering
	\includegraphics[width=0.7\linewidth]{images/FondoNegro.png}
	\caption{Pixeles defectuosos del escáner }
	\label{fig:fondoNegro}
\end{figure}

Para corregir parte de este ruido en las posteriores tomas se escaneo cada imagen múltiples veces, lo que reduce la cantidad de pixeles defectuosos.\\

Para continuar realizando pruebas en los métodos usados se analiza el cuadrado de 5 Gy de 10x10 cm que se irradió. La película resultante se muestra en la figura \ref{fig:cuadrado5Gy}. En este se muestran los puntos fiduciales que se usan para identificar el eje del campo.\\

\begin{figure}[H]
	\centering
	\includegraphics[width=0.7\linewidth]{images/peliculaCuadrado.png}
	\caption{Película irradiada con cuadrado de 5 Gy }
	\label{fig:cuadrado5Gy}
\end{figure}

Este cuadrado sirve para identificar fácilmente las ventajas del método multicanal con respecto a la calibración mediante canales individuales. En la figura \ref{fig:MapaCuadrado} se muestra el mapa de dosis del cuadrado obtenido junto con la sepración que este método realiza de las partes de la transmitancia que son independientes de la dosis que se absorbió.\\
\begin{figure}[H]
	\centering
	\subfloat[Mapa de dosis del cuadrado de 5 Gy]{\includegraphics[width=0.5\textwidth]{images/mapaCuadradoConMulticanal.png}\label{fig:MapaCuadradoMulti}}
	\hfill
	\subfloat[Defectos de la película obtenidos con el método multicanal]{\includegraphics[width=0.8\textwidth]{images/imperfeccionesPeliculaCuadrado.png}\label{fig:fondoCuadrado5Gy}}
	\caption{Mapa de dosis obtenido con el método multicanal}
	\label{fig:MapaCuadrado}
\end{figure}

En el mapa de defectos se encuentran tanto los puntos fiduciales, como las barras transportadoras propias del escáner. Estas componentes ya no son tenidas en cuenta en el calculo de dosis de cada punto. Esto resulta finalmente en un mapa de dosis más limpio, libre de una gran parte de irregularidades y efectos asociados a inhomogeneidades de la película.\\

Podemos comparar los perfiles de dosis sobre una linea que pasa por centro de cuadrado obtenidos con canales individuales y usando el método multicanal. En la figura \ref{fig:perfilesMapaCuadrado} se muestran perfiles centrales obtenidos con cada método. Se observa cómo se reduce el rudio con el método multicanal, resultando en un medida fiable de la dosis obtenida en cada punto. 
\begin{figure}[H]
	\centering
	\subfloat[Perfil de dosis central con método de un solo canal]{\includegraphics[width=0.7\textwidth]{images/perfilDosisCuadradoUnoSolo.png}\label{fig:perfilSolo}}
	\hfill
	\subfloat[Perfil de dosis central con método multicanal ]{\includegraphics[width=0.7\textwidth]{images/perfilDosisCuadradoMulticanal.png}\label{fig:perfilMultiple}}
	\caption{Perfil central de dosis en película con cuadrado de 5 Gy}
	\label{fig:perfilesMapaCuadrado}
\end{figure}

Otra manera de observar la superioridad del método multicanal sobre el método de un solo canal es observar el mismo perfil, que se muestra en la figura \ref{fig:perfilCero}, sobre la película sin irradiar como en la figura \ref{fig:perhor} pero ahora en el mapa calculado con la calibración multicanal. Aquí se evidencia también que este método es útil para corregir la sobreestimación debida a la cercanía al borde del escáner.\\

\begin{figure}[H]
	\centering
	\includegraphics[width=0.7\linewidth]{images/perfilHorizontalDeDosisCeroMulticanal.png}
	\caption{Perfil de dosis horizontal en película sin irradiar con método multicanal }
	\label{fig:perfilCero}
\end{figure}

También se produjo la separación de las componentes independientes de dosis en la película sin irradiar, las cuales se muestran en la figura \ref{fig:irregularCero} 
\begin{figure}[H]
	\centering
	\includegraphics[width=0.7\linewidth]{images/fondoIndependienteDosisPeliculaCero.png}
	\caption{Irregularidades en película sin escanear obtenidas con método multicanal }
	\label{fig:irregularCero}
\end{figure}

Por otro lado, también se investigó la diferencia en ruido al usar canales de color de 8 o 16 bits. En la figura  \ref{fig:8o16} se muestran perfiles centrales para el cuadrado de 5 Gy cuando la película se escanea en modo de 8 y 16 bits. Aquí se evidencia que usando 8 bits se presentan mayores diferencias entre las dosis calculadas en cada canal debido a la menor capacidad de diferenciar colores. Se encontró que, contrario a lo que se esperaba, usar 8 bits de profundidad por canal de color presenta más ruido en la determinación de la dosis que usar canales de 16 bits.\\
\begin{figure}[H]
	\centering
	\subfloat[Perfil de dosis central con profundidad de 8 bits]{\includegraphics[width=0.7\textwidth]{images/perfilCuadradoMenosBit.png}\label{fig:perfil8}}
	\hfill
	\subfloat[Perfil de dosis central con profundidad de 16 bits ]{\includegraphics[width=0.7\textwidth]{images/perfilDosisCuadradoUnoSolo.png}\label{fig:perfil16}}
	\caption{Perfil central de dosis en película con cuadrado de 5 Gy a 8 y 16 bits de profundidad de color}
	\label{fig:8o16}
\end{figure}

Finalmente, se comprobó la dependencia de la transmitancia mediada en el escáner dependiendo de la orientación de la película con respecto a la lampara de escaneo, provocada por la polarización que ocurre en la medida. Este efecto se ve en las curvas de calibración presentadas en la figura \ref{fig:efectoOrientacion}, en las cuales se evidencia un desplazamiento en las transmitancias promedio cuando se rota la misma película 90 grados en la misma posición de escaneo.\\
\begin{figure}[H]
	\centering
	\subfloat[Curva de calibración con películas sin rotar]{\includegraphics[width=0.7\textwidth]{images/calibracion0-20-sinRotar.png}\label{fig:sinRotar}}
	\hfill
	\subfloat[Curvas de calibración con películas rotadas 90 grados ]{\includegraphics[width=0.7\textwidth]{images/calibracion0-20-Rotadas.png}\label{fig:perfil16}}
	\caption{Efecto de la orientación de escaneo sobre las transmitancias medidas}
	\label{fig:efectoOrientacion}
\end{figure}
Este efecto no es significativo en las medidas tomadas siempre y cuando se procuren orientar las películas cada vez en la misma dirección con respecto a la lampara del escáner.\\

No fue posible comprobar el efecto de otras variables como la temperatura y el tiempo post-exposición dadas las restricciones de tiempo y movilidad que se presentaron en el transcurso del trabajo, además de la poca capacidad de control que se tiene sobre estas variables en el transcurso del proceso.\\

\section{Mapas de dosis}


El plan piramidal de prueba produjo una película resultante que se ilustra en la figura \\
\begin{figure}
	\centering
	\missingfigure[figwidth=\linewidth,figcolor=white]{Plan de tratamiento}
	
	\caption{Pelicula EBT3 irradiada con plan de piramide }
	\label{fig:piramideEscaneada}
\end{figure}

En la figura \ref{fig:mapaPiramide} se muestra el mapa de dosis calculado con la digitalización anterior, junto al mapa de dosis que calculó el sistema de planeación en el mismo plano. Para alinear estos planes se usaron cuatro puntos fiduciales y un proceso de registro usando la librería SimpleITK de python.\\

\begin{figure}
	\centering
	\missingfigure[figwidth=\linewidth,figcolor=white]{Plan de tratamiento piramide comparada}
	
	\caption{Mapa de dosis calculado experimental y computacionalmente }
	\label{fig:mapaPiramide}
\end{figure}

Se pueden obtener diversas estadísticas para comparar estos planes, por ejemplo, en la figura \ref{fig:perfilesDosisPiramide} se muestran perfiles de dosis a diferentes niveles que muestran una concordancia entre las dosis calculadas de ambas maneras.\\
\begin{figure}
	\centering
	\missingfigure[figwidth=\linewidth,figcolor=white]{Plan de tratamiento piramide comparada}
	
	\caption{Perfiles de dosis en diferentes planos }
	\label{fig:perfilesDosisPiramide}
\end{figure}
Igualmente, en la figura \ref{fig:histogramasDosisPiramide} se muestran comparaciones de los histogramas de dosis obtenidos.\\
\begin{figure}
	\centering
	\missingfigure[figwidth=\linewidth,figcolor=white]{Plan de tratamiento piramide comparada histograma}
	
	\caption{Histogramas de dosis para plan pirámide }
	\label{fig:histogramasDosisPiramide}
\end{figure}

Finalmente, en la figura \ref{fig:isodosisPiramide}  se presentan curvas de isodosis de estos mapas.\\ 
\begin{figure}
	\centering
	\missingfigure[figwidth=\linewidth,figcolor=white]{Plan de tratamiento piramide comparada isodosis}
	
	\caption{Curvas de isodosis para plan pirámide }
	\label{fig:isodosisPiramide}
\end{figure}

Similarmente, se realiza un análisis comparativo del plan de tratamiento de mama 
\begin{figure}
	\centering
	\missingfigure[figwidth=\linewidth,figcolor=white]{Plan de tratamiento mama}
	
	\caption{Plan de tratamiento de mama escaneado }
	\label{fig:mamaEscaneada}
\end{figure}
\begin{figure}
	\centering
	\missingfigure[figwidth=\linewidth,figcolor=white]{Mapas de dosis mama}
	
	\caption{Mapa de dosis para plan de mama }
	\label{fig:mapaMama}
\end{figure}
\begin{figure}
	\centering
	\missingfigure[figwidth=\linewidth,figcolor=white]{Plan de tratamiento piramide comparada isodosis}
	
	\caption{Curvas de isodosis para plan pirámide }
	\label{fig:histogramaMama}
\end{figure}
\begin{figure}
	\centering
	\missingfigure[figwidth=\linewidth,figcolor=white]{Plan de tratamiento piramide comparada isodosis}
	
	\caption{Curvas de isodosis para plan pirámide }
	\label{fig:isodosisMama}
\end{figure}

\section{Comparaciones $\Gamma$}

En el caso del plan pirámide, se obtiene la matriz $\Gamma$ mostrada en la figura \ref{fig:gammaPiramide}, así como el histograma de $\Gamma$ que evalúa la correspondencia general de los pixeles en ambas distribuciones.\\
\begin{figure}
	\centering
	\missingfigure[figwidth=\linewidth,figcolor=white]{Gamma Piramide}
	\caption{Análisis $\Gamma$ para plan piramide }
	\label{fig:gammaPiramide}
\end{figure}

Finalmente, en el caso del plan de tratamiento se mama, se muestra el mismo análisis en la figura \ref{fig:gammaMama}
\begin{figure}
	\centering
	\missingfigure[figwidth=\linewidth,figcolor=white]{Gamma mama}
	\caption{Análisis $\Gamma$ para plan de tratamiento de mama}
	\label{fig:gammaMama}
\end{figure}




